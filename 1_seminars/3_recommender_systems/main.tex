\documentclass[a4paper]{article} 

\usepackage{graphicx}
\graphicspath{ {./images/} }
\usepackage{wrapfig}

\input{head}
\begin{document}

%-------------------------------
%	TITLE SECTION
%-------------------------------

\fancyhead[C]{}
\hrule \medskip % Upper rule
\begin{minipage}{0.24\textwidth} 
\raggedright
\footnotesize
Quentin Le Roux \hfill
\end{minipage}
\begin{minipage}{0.5\textwidth} 
\centering 

Seminar 3 -- Recommender Systems -- 1-page Summary
\end{minipage}
\begin{minipage}{0.245\textwidth} 
\raggedleft
\today
\end{minipage}
\hrule 
\bigskip

%-------------------------------
%	CONTENTS
%-------------------------------

The seminar covered the topic of \textbf{Recommender Systems: an industrial perspective}, presented by Boris Shminke from CNRS (Laboratoire J.A. Dieudonné).

\section{What are recommender systems (RS)?}

A RS \textit{returns a curated collection of item to a user based on a context and their request(s) while learning in an iterative from their choices and preferences}. A specific type is considered here: "\textbf{item2user}".

RS differ from search systems as their responses are not ranked [lists]. RS offer no ideal ordering and recommendations are personally tailored to the user. RS are also not considered advertising as a user has implicitly agreed to consume content. The goal being to show related, consumable content. User preferences can either be \textit{explicit} (e.g. reviews, stars) or \textit{implicit} (e.g. viewing, purchasing).

Mentioned above, RS work within a context: a set of user-independent factors that will nonetheless influence the recommendation results (e.g. time, which can be hierarchically split further such as hour-day-season). 

%How to build a RS: collect preferences of users for items. predict unknown preferences. profit

\section{Matrix Completion Problem}

Matrix Completion is one way to answer the need for RS as it is a way to represent input data: 
\begin{quote}
Given a list of $m$ users ${u_1, ..., u_n}$ and $n$ items ${i_1, ..., i_n}$, the user preferences toward the items can be represented as a matrix of dimensions $m \times n$ matrix $A$, where each entry is either a rating or incomplete.
\end{quote}
A bilinear model can be applied to fill the matrix (i.e. finding the user’s preference for an item) by applying: $$r_{u, i} = \langle x^u, y^i \rangle= \underset{j=1}{\overset{K}{\sum}x_j^uy_j^i} \Leftrightarrow R = XY^T$$ with $x^u$ a representation of a user's preferences and $y^i$ a representation of the preferences attributed to an item, both in a K-dimensional latent space.
Such a technique helps compute a user preferences for each single item and find recommendations for users. Other techniques such as Singular Value Decomposition (SVD) can be used to ease the process by performing dimensionality reduction.
\\\\
\textbf{Remark}: [B. Shminke] "Bilinear models are a king of neural network."

\section{Recommendation Engine Architecture, i.e. How to Build One?}
The construction of a RS involves a three-stage architecture:

\subsection{Candidate Selection}
Candidate selection is performed on a sizeable number of items (c. $10^6$ to $10^9$ items). This phase involves clustering (with matrix factorisation or KNN) to cast users and items in a lower dimensional latent space.

\subsection{Ranking}
Neural Networks are preferred for their task effectiveness. This phase tries to bridge both technical and business needs. Indeed, the RS is trained to correspond to multiple business-centered criteria (e.g. promoting new releases, following trends, etc.) as well as meeting concepts of coverage, diversity, etc. 
\begin{quote}
Precision alone is not sufficient!
\end{quote}
\subsection{Re-Ranking}
Re-ranking corresponds to a phase where trade-off are made (and new training periods can be run) between losing precision in order to gain a better handling on other criteria as mentioned above. It can be considered fine-tuning.
\\\\
In conclusion, a given example of RS building is as such: selecting candidates with SVD, ranking recommendations with a two-tower neural network, re-ranking for improving diversity of recommendations.
\end{document}

%doc by Quentin Le Roux