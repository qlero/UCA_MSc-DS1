\documentclass[twocolumn, letterpaper]{scrartcl}

\usepackage{uog_factsheet}
\usepackage{xcolor}
\usepackage{hyperref}

\definecolor{seablue}{RGB}{0,127,169}

\begin{document}
    \title{\color{seablue}User Consent \& Privacy Policies Project}

	\maketitle
	
    % \section*{Instructions}
    
    % \begin{itemize}
    %     \item Open history of your browser, choose 3 websites from your browsing history, open these 3 website in a private/incognito window
    %     \item Is there a consent banner? Does it comply with elements of valid consent (Free, specific, informed, unambiguous)?
    %     \item Check the slides to validate your analysis, write a 2-page report with your analysis.
    % \end{itemize}
    
    \section*{Website Choices}
        
         For the sake of diversity of industries, three websites were selected: \textbf{Games Workshop}\cite{GW} (the world's largest miniatures game retailer), \textbf{Le Monde}\cite{LM} (A French newspaper) and \textbf{LinkedIn}\cite{LD} (the world's largest professional social media).
	
	\section{Analysis of the websites' policies and handling of consent}
	
	    We recall that consent forms are a mechanism to give data subjects control and choice over whether or not their personal data will be collected and processed. It must be given before any processing starts and is only valid when four criteria based on the General Data Protection Regulation's article 4(11) are met: i) \textbf{Free}, ii) \textbf{Specific}, iii) \textbf{Informed}, and iv) \textbf{Unambiguous}.
	
    \subsection*{Games Workshop's}

	    Games Workshop's cookie policy provides an \textbf{example of a policy that is neither free, specific, informed or unambiguous}.
	
    	\textbf{The consent policy is not free}. The data subjects have no choice when presented with the consent banner (see Fig. \ref{fig:a}). It is more of a warning than a prompt: "\textit{This website uses cookies to personalise content and advertising, and to analyse our traffic. By continuing to use this site you are agreeing to our use of cookies}". We could qualify this instance as an \textbf{imbalance of power} as coercion is involved -- the data subjects have no choice but to click on the \textit{continue} button. This is also an example of \textbf{conditionality} and \textbf{detrimentality} (See Fig. \ref{fig:b}). To opt-out, the data subjects must personally set up their browsers to restrict cookies\footnote{Games Workshop does not provide data subjects ways to restrict the use of cookies. Instead, they are advised to personally parameter their browser via the use of third-parties plug-ins to opt out of systems such as Google AdSense.} at the risk of experiencing a loss in website functionalities. The right to withdraw consent (as per GDPR Article 7(3)) is not provided on the website.
    	
    	\textbf{The consent policy is not specific}. The stated purpose of data usage is too broad, mentioning on the front page banner that cookies would be used "\textit{to personalise content and advertising, and to analyse our traffic}". Some detail is provided in their Cookie Policy page\cite{GW_CN} but the descriptions are not extensive (e.g. there is no indication as to how long data is kept). A \textbf{lack of granularity} in consent requests is found here, confirming the lack of choice exposed in the first paragraph.
    	
    	Given what we saw, \textbf{the consent policy does not allow informed consent}. The language used and the instructions given to opt out of tracking are not intended for lay people, as shown in Footnote 1. We will insist on \textbf{the consent banner presenting an example of ambiguous consent}. The only possible action is to "continue" and close the cookie banner.
    	
    	To sum up, Games Workshop has a lackluster implementation of a user consent policy in the context of GDPR. The company fails to meet all four criteria that would indicate data subjects can provide a valid consent. Of note, the Cookie Policy page mentions that the website uses a technology called Web Beacon, which is not fully stated in the main page's consent bar and which also cannot be opted out of.
 	
 		\begin{figure}
    % 		Figure 1\label{fig:a}. Games Workshop's Main Page
    		\includegraphics[width=0.95\linewidth]{gw_website.JPG}
     		\caption{Games Workshop's Cookie Banner \label{fig:a}}
     	\end{figure}
    
        \begin{figure}[tbp]	
            % Figure 2\label{fig:b}. Example of conditionality in Games Workshop's policy
            \includegraphics[width=0.95\linewidth]{conditionality.JPG}
            \caption{Example of Conditionality in Games Workshop's Cookie Policy Page \label{fig:b}}
        \end{figure}
        
	\subsection*{Le Monde's}

    	Le Monde's website provides elements that favor the interpretation that data subjects' consent is \textbf{free}, \textbf{specific}, \textbf{informed}, and \textbf{unambiguous}. However, we will present two caveats to this. 
        
    	Overall, there is no apparent evidence of imbalances of power, conditionality or detrimentality. It implies that consent is freely given (e.g. Consent is asked in the scope of providing news content to the data subjects). Specificity is present as the purpose of collecting data is clearly stated and data subjects can opt-in to cookie tracking at a granular level. All the same, tracking options are ticked off by default in the Cookie Parameter page. The possibility to refuse all cookies, besides those legally-allowed to be non-opt-out, is available as well (See Fig. \ref{fig:c}). Finally, consent seems to be informed as the purposes of data usage is clearly stated on the front page (See Fig. \ref{fig:d}) and an exhaustive description of the collected data is detailed in the parameter page. There is no apparent ambiguity.
    	
    	\begin{figure}[tbp]	
            % Figure 3\label{fig:c}. Le Monde's Main Page 
            \includegraphics[width=0.9\linewidth]{lm_cn.JPG}
            \caption{Le Monde's Cookie Parameter Page \label{fig:c}}
        \end{figure}
        
    	However, the validity of the data subjects' consent can be put into question as, were they to refuse all cookies, Le Monde's website would constantly display a irremovable ad (for their subscription service) that is obstructing on some devices (See Fig. \ref{fig:e}). \textbf{This could unmake the argument that Le Monde's policy fulfills the 'free' criterion of valid consent as stated by Article 4(11) of GDPR}. This irremovable ad could be considered a detriment and a consequence of not consenting. On the other hand, \textit{there is no prior indication that this bar would appear if the cookies were deactivated}. This bar is not used as a threat to the data subjects' user experience. In the end, it might only be an issue on some devices for which the website was not optimized for. 
    	
    	\begin{figure}[tbp]
            % Figure 4\label{fig:d}. Le Monde's Cookie Policy Parameter Page 
            \includegraphics[width=0.9\linewidth]{lm_website.JPG}
            \caption{Le Monde's Cookie Banner \label{fig:d}}
        \end{figure}
        
        \begin{figure}[tbp]
            % Figure 5\label{fig:e}. Le Monde's Unremovable Ad Bar
            \includegraphics[width=0.95\linewidth]{lm_sub.JPG}
            \caption{Le Monde's Unremovable Ad \label{fig:e}}
        \end{figure}
        
    	Another weak point relates to accessing Le Monde's Cookies and Privacy Policy page\cite{LM2}. It happens to be \textbf{unreadable if the data subjects have not yet accepted the policy}, i.e. there is no way for data subjects to read the detailed policy (which includes a table detailing the cookies used by Le Monde, the collected and shared data, and an associated purpose and expiration date) \textbf{without accepting the cookies first} (See Fig. \ref{fig:f}). This implies that there might not be an informed consent on Le Monde's website, and Recital\$42 and Article 7(3) of GDPR might not be fully satisfied. 
        
        \begin{figure}[tbp]	
            % Figure 6\label{fig:f}. Le Monde's Unreadable Detailed Cookie Policy
            \includegraphics[width=0.95\linewidth]{lm_policy.JPG}
            \caption{Le Monde's Cookie Policy Page\label{fig:f}}
        \end{figure}
        
        \begin{figure}[tbp]	
        % Figure 6\label{fig:g}. LinkedIn's Main Page 
        \includegraphics[width=0.95\linewidth]{ld_website.JPG}
        \caption{LinkedIn's Cookie Banner \label{fig:g}}
        \end{figure}
        
	\subsection*{LinkedIn's}
	    
	    Similar to Le Monde, LinkedIn provides elements that favor the interpretation that consent given by data subjects is \textbf{free}, \textbf{specific}, \textbf{informed}, and \textbf{unambiguous}. For instance, LinkedIn provides a detailed consent bar (See Fig. \ref{fig:g}) as well as a dedicated cookie page where data subjects can decide to opt-in (See Fig. \ref{fig:h}).
        
	    As with Le Monde, LinkedIn provides a detailed Cookie Policy page\cite{LI_CN} (See Fig. \ref{fig:i}) that contains a table describing the type of cookies used, the data collected and shared, and an associated purpose and expiration date (LinkedIn does not hide that page behind a pop-up bar like Le Monde).
	    
	    LinkedIn's weak point is that, were the data subjects to leave the Cookie Parameter page without inputing any modification, the consent bar would never appear again. i.e., once the cookie parameters have been opened, the website considers them modified and accepted as-is by the data subjects. Afterward, finding the parameter page again is complex. This implies \textbf{withdrawal of consent} can be hard to perform.

        \begin{figure}[tbp]	
        \includegraphics[width=0.9\linewidth]{ld_cn.JPG}
        \caption{LinkedIn's Cookie Parameter Page \label{fig:h}}
        \end{figure}
        
        \begin{figure}[tbp]	
        \includegraphics[width=0.9\linewidth]{ld_table.JPG}
        \caption{LinkedIn's Cookie Policy Page \label{fig:i}}
        \end{figure}
        
	\subsection*{Conclusion}
	
	Out of all three websites, Games Workshop's is the most at fault in how it handles consent. It fails in all four validity categories set by the Article 4(11) of GDPR.
	
	Le Monde and LinkedIn's websites are much more compliant with the article, scoring in all four categories. However, both websites present edge cases that may cast doubts on the validity of data subjects' consent on their platforms.
	
    \bibliographystyle{unsrtnat}   
    \begin{thebibliography}{9}

    \bibitem{GW}
    	Games Workshop Limited, \textit{\href{https://www.games-workshop.com/en-US/Home}{website}}.
    \bibitem{LM}
    	Le Monde SA, \textit{\href{https://www.lemonde.fr/}{website}}.
    \bibitem{LD}
    	LinkedIn Corporation, \textit{\href{https://www.linkedin.com/}{website}}.
    \bibitem{GW_CN}
    	Games Workshop's Cookie Policy page, \textit{\href{https://www.games-workshop.com/en-EU/Cookie-Notice}{website}}.
    \bibitem{LM2}
    	Le Monde's Cookie Policy Page, \textit{\href{https://www.lemonde.fr/confidentialite/}{website}}.
    \bibitem{LI_CN}
    	LinkedIn's Cookie Policy Page, \textit{\href{https://www.linkedin.com/legal/cookie-policy}{website}}, \textit{\href{https://www.linkedin.com/legal/l/cookie-table}{table}}.
    	
    \end{thebibliography}

\end{document}